
\documentclass[letterpaper,11pt]{article}

\usepackage{latexsym}
\usepackage[empty]{fullpage}
\usepackage{titlesec}
\usepackage{marvosym}
\usepackage[usenames,dvipsnames]{color}
\usepackage{verbatim}
\usepackage{enumitem}
\usepackage[hidelinks]{hyperref}
\usepackage{fancyhdr}
\usepackage[english]{babel}
\usepackage{tabularx}
\usepackage{fontawesome5}
\usepackage{multicol}
\setlength{\multicolsep}{-3.0pt}
\setlength{\columnsep}{-1pt}
\input{glyphtounicode}


%----------FONT OPTIONS----------
% sans-serif
% \usepackage[sfdefault]{FiraSans}
% \usepackage[sfdefault]{roboto}
% \usepackage[sfdefault]{noto-sans}
% \usepackage[default]{sourcesanspro}

% serif
% \usepackage{CormorantGaramond}
% \usepackage{charter}


\pagestyle{fancy}
\fancyhf{} % clear all header and footer fields
\fancyfoot{}
\renewcommand{\headrulewidth}{0pt}
\renewcommand{\footrulewidth}{0pt}

% Adjust margins
\addtolength{\oddsidemargin}{-0.6in}
\addtolength{\evensidemargin}{-0.5in}
\addtolength{\textwidth}{1.19in}
\addtolength{\topmargin}{-.7in}
\addtolength{\textheight}{1.4in}

\urlstyle{same}

\raggedbottom
\raggedright
\setlength{\tabcolsep}{0in}

% Sections formatting
\titleformat{\section}{
  \vspace{-4pt}\scshape\raggedright\large\bfseries
}{}{0em}{}[\color{black}\titlerule \vspace{-5pt}]

% Ensure that generate pdf is machine readable/ATS parsable
\pdfgentounicode=1

%-------------------------
% Custom commands
\newcommand{\resumeItem}[1]{
  \item\small{
    {#1 \vspace{-2pt}}
  }
}

\newcommand{\classesList}[4]{
    \item\small{
        {#1 #2 #3 #4 \vspace{-2pt}}
  }
}

\newcommand{\resumeSubheading}[4]{
  \vspace{-2pt}\item
    \begin{tabular*}{1.0\textwidth}[t]{l@{\extracolsep{\fill}}r}
      \textbf{#1} & \textbf{\small #2} \\
      \textit{\small#3} & \textit{\small #4} \\
    \end{tabular*}\vspace{-7pt}
}

\newcommand{\resumeSubSubheading}[2]{
    \item
    \begin{tabular*}{0.97\textwidth}{l@{\extracolsep{\fill}}r}
      \textit{\small#1} & \textit{\small #2} \\
    \end{tabular*}\vspace{-7pt}
}

\newcommand{\resumeProjectHeading}[2]{
    \item
    \begin{tabular*}{1.001\textwidth}{l@{\extracolsep{\fill}}r}
      \small#1 & \textbf{\small #2}\\
    \end{tabular*}\vspace{-7pt}
}

\newcommand{\resumeSubItem}[1]{\resumeItem{#1}\vspace{-4pt}}

\renewcommand\labelitemi{$\vcenter{\hbox{\tiny$\bullet$}}$}
\renewcommand\labelitemii{$\vcenter{\hbox{\tiny$\bullet$}}$}

\newcommand{\resumeSubHeadingListStart}{\begin{itemize}[leftmargin=0.0in, label={}]}
\newcommand{\resumeSubHeadingListEnd}{\end{itemize}}
\newcommand{\resumeItemListStart}{\begin{itemize}}
\newcommand{\resumeItemListEnd}{\end{itemize}\vspace{-5pt}}

%-------------------------------------------
%%%%%%  RESUME STARTS HERE  %%%%%%%%%%%%%%%%%%%%%%%%%%%%


\begin{document}

%----------HEADING----------
% \begin{tabular*}{\textwidth}{l@{\extracolsep{\fill}}r}
%   \textbf{\href{http://sourabhbajaj.com/}{\Large Sourabh Bajaj}} & Email : \href{mailto:sourabh@sourabhbajaj.com}{sourabh@sourabhbajaj.com}\\
%   \href{http://sourabhbajaj.com/}{http://www.sourabhbajaj.com} & Mobile : +1-123-456-7890 \\
% \end{tabular*}

%----------HEADING----------
\begin{center}
    {\Huge \scshape Chika Maduabuchi} \\ \vspace{1pt}
    Cambridge 02139, MA, USA \\ \vspace{1pt}
    \small \href{mailto:chikap421@gmail.com}{\raisebox{-0.2\height}\faEnvelope\  \underline{chikap421@gmail.com}} ~ 
    \href{https://github.com/chikap421}{\raisebox{-0.2\height}\faGithub\ \underline{github.com/chikap421}} ~
    \href{https://www.linkedin.com/in/chika-maduabuchi-4796701b1/}{\raisebox{-0.2\height}\faLinkedin\ \underline{linkedin.com/in/chika-maduabuchi-4796701b1}} ~
    \href{tel:+16178218696}{\raisebox{-0.2\height}\faPhone\ +1 617 821 8696}
    \vspace{-8pt}
\end{center}



%-----------EDUCATION-----------
\section{Education}
  \resumeSubHeadingListStart
    \resumeSubheading
      {Massachusetts Institute of Technology}{2022 -- 2024}
      {Master of Science in Nuclear Science and Engineering}{Cambridge, MA}
      \resumeItemListStart
        \resumeItem{Awarded \$91,000 research funding from Mathworks for developing computer vision models for autonomous boiling.}
      \resumeItemListEnd
    \resumeSubheading
      {University of Nigeria Nsukka}{2014 -- 2019}
      {Bachelor of Engineering in Mechanical Engineering}{Nsukka, EN}
  \resumeSubHeadingListEnd

%------RELEVANT COURSEWORK-------
\section{Relevant Coursework}
    \begin{multicols}{3}
        \begin{itemize}[itemsep=-5pt, parsep=3pt]
            \item\small Advanced Machine Learning
            \item Modelling with Machine Learning
            \item Advanced Numerical Methods
            \item Advanced Calculus
            \item Computer Programming
            \item Complex Analysis
            \item Probability and Statistics
            \item Advanced Linear Algebra
            \item  Computational Algorithms
        \end{itemize}
    \end{multicols}
    \vspace*{2.0\multicolsep}

%------RESEARCH INTERESTS-------
\section{Research Interests}
    \begin{multicols}{2}
        \begin{itemize}[itemsep=-5pt, parsep=3pt]
            \item\small Machine Learning
            \item Computer Vision
            \item Natural Language Processing
            \item Generative AI
            \item Foundation Models
            \item Reinforcement Learning
        \end{itemize}
    \end{multicols}
    \vspace*{2.0\multicolsep}

%-----------EXPERIENCE-----------
\section{Experience}
  \resumeSubHeadingListStart

    \resumeSubheading
      {MIT Red Lab}{September 2022 -- Present}
      {Research Fellow}{Cambridge, MA}
      \resumeItemListStart
        \resumeItem{Implemented transfer learning techniques using U-Net CNNs from \href{https://arxiv.org/abs/1505.04597}{\underline{pre-trained biological cell models}} to boiling videos for accurate segmentation (over 98\% Dice coefficients and Jaccard index) of bubbles to enable autonomous systems control.}
        \resumeItem{Fine-tuned state-of-the-art vision transformer (ViT-H) foundation model, \href{https://arxiv.org/abs/2304.02643}{\underline{META SAM}}, on PyTorch to enhance generalization capabilities on new boiling images with complex patterns.}
        \resumeItem{Quantified uncertainty in the vision model predictions by implementing a randomized sorting algorithm which evaluated the error on the boundaries of the model predictions.}
        \resumeItem{Attended weekly group meetings and met my research advisor twice a week to share research updates and receive scholarly feedback.}
      \resumeItemListEnd

    \resumeSubheading
      {Artificial Intelligence Laboratory}{October 2021 -- August 2022}
      {Research Lead}{Nsukka, EN}
      \resumeItemListStart
            \resumeItem{Led a team of data scientists to implement neural networks, support vector machines, boosting and bagging ML models in scikit-learn to forecast global weather parameters and solar power in strategic cities across the 7 continents.}
        \resumeItem{Conducted exploratory data analysis (EDA) and recursive feature elimination (RFE) on the 2 years time series data to determine the most relevant features for ML forecast of the system power and efficiency. }
        \resumeItem{Tuned the models to give the best prediction output based on numerous regression performance metrics including the relative root mean squared error (RRMSE) and Pearson correlation coefficient ($R^2$)}
        \resumeItem{Validated the ML forecasts with year ahead ground-truth data from NASA and submitted the developed manuscript for publication consideration in \href{https://www.sciencedirect.com/journal/engineering-applications-of-artificial-intelligence}{\underline{Engineering Applications of Artificial Intelligence}}.}
    \resumeItemListEnd
    
  \resumeSubHeadingListEnd
\vspace{-16pt}

%-----------PROJECTS-----------
\section{Projects}
    \vspace{-5pt}
    \resumeSubHeadingListStart
      \resumeProjectHeading
          {\textbf{Generative AI - Diffusion Models} $|$ \emph{Python, PyTorch, CNNs, Einops}}{October 2023}
          \resumeItemListStart
            \resumeItem{Developed a PyTorch-based implementation of Denoising Diffusion Probabilistic Models (DDPMs) to generate high-quality samples from noise through a reverse noising process as an implementation of the paper by \href{https://arxiv.org/abs/2006.11239}{\underline{Ho et al.}} }
            \resumeItem{Designed a custom U-Net CNN architecture with sinusoidal positional embedding to condition the denoising process on the diffusion timestep for effective image denoising on MNIST and Fashion-MNIST datasets.}
            \resumeItem{Utilized a sophisticated training loop with noise application at random timesteps and hyperparameter optimization to ensure accurate noise prediction and efficient learning.}
            \resumeItem{Created dynamic visualization tools for the diffusion process and model evaluation, offering intuitive insights into the model's generative capabilities and performance.}
          \resumeItemListEnd
          \vspace{-13pt}
      \resumeProjectHeading
          {\textbf{Reinforcement Learning  - Policy Gradients} $|$ \emph{Python, PyTorch, gym-minigrid, ACModel}}{November 2023}
          \resumeItemListStart
            \resumeItem{Developed and implemented the REINFORCE algorithm using PyTorch by training the model to solve a door-key task in a custom MiniGrid environment.}
            
          \resumeItem{Designed and implemented an Actor-Critic model with convolutional neural networks for feature extraction and policy and value function approximation in PyTorch.}
            \resumeItem{Conducted experiments with both REINFORCE and Vanilla Policy Gradient methods, incorporating baselines to reduce variance and improve learning stability through hyperparameter tuning.}
            \resumeItem{Created utility functions for preprocessing observations, collecting experiences, and visualizing model performance, along with integrating \texttt{sensorimotor-checker} for testing algorithm correctness.}
          \resumeItemListEnd 
          \vspace{-13pt}
          \resumeProjectHeading
          {\textbf{Character-Level Language Modeling - GPT Architecture} $|$ \emph{Python, PyTorch, Transformers}}{December 2023}
          \resumeItemListStart
            \resumeItem{Engineered a character-level text generation model using PyTorch, encapsulating the core principles of Transformer architectures, to autonomously generate text in the style of Shakespeare with over 0.209 million parameters.}
            \resumeItem{Implemented and optimized a simplified GPT-like architecture, including self-attention mechanisms, multi-head attention, and feedforward networks, tailored for understanding and generating Shakespearean text.}
            \resumeItem{Employed character-to-index and index-to-character mappings ($stoi$ and $itos$) for efficient text encoding and decoding, facilitating the model's learning process from the Tiny Shakespeare dataset through supervised learning techniques.}
            \resumeItem{Demonstrated the model's generative capabilities by training it to predict subsequent characters, achieving coherent text generation, and showcased through a custom text generation function post-training, the model’s ability to synthesize new textual content stylistically similar to the training dataset.}
          \resumeItemListEnd 
    \resumeSubHeadingListEnd
\vspace{-15pt}


%
%-----------PROGRAMMING SKILLS-----------
\section{Technical Skills}
 \begin{itemize}[leftmargin=0.15in, label={}]
    \small{\item{
     \textbf{Machine Learning Libraries}{: PyTorch, scikit-learn, OpenCV, TensorFlow} \\
     \textbf{Languages}{: Python, C++, JavaScript, HTML/CSS, MATLAB, R, SQL} \\
     \textbf{Developer Tools}{: VS Code, Jupyter Notebook, Google Colab, CUDA, ImageJ/Fiji} \\
     \textbf{Technologies/Frameworks}{: Linux, GitHub, Flask, AWS EC2} \\
    }}
 \end{itemize}
 \vspace{-16pt}


%-----------INVOLVEMENT---------------
\section{Leadership / Extracurricular}
    \resumeSubHeadingListStart
        \resumeSubheading{Enrai}{Summer 2022 -- Present}{President}{University of Nigeria}
            \resumeItemListStart
                \resumeItem{Founded and lead Enrai, an organization dedicated to providing research experience in computer programming and optimization, focusing on solving societal problems and contributing to open-source projects.}
                \resumeItem{Direct and mentor a diverse team of 10+ members, fostering a collaborative environment to enhance skills in advanced computational techniques and real-world problem-solving.}
                \resumeItem{Successfully guided three members through comprehensive research projects and skill development, resulting in their acceptance into fully-funded Ph.D. programs at prestigious universities in the United States.}
            \resumeItemListEnd
        
    \resumeSubHeadingListEnd


\end{document}
